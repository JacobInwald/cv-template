%%%%%%%%%%%%%%%%%%%%%%%%%%%%%%%%%%%%%%%%%
% Developer CV
% LaTeX Class
% Version 2.0 (12/10/23)
%
% This class originates from:
% http://www.LaTeXTemplates.com
%
% Authors:
% Omar Roldan
% Based on a template by  Jan Vorisek (jan@vorisek.me)
% Based on a template by Jan Küster (info@jankuester.com)
% Modified for LaTeX Templates by Vel (vel@LaTeXTemplates.com)
%
% License:
% The MIT License (see included LICENSE file)
%
%%%%%%%%%%%%%%%%%%%%%%%%%%%%%%%%%%%%%%%%%

%----------------------------------------------------------------------------------------
%	PACKAGES AND OTHER DOCUMENT CONFIGURATIONS
%----------------------------------------------------------------------------------------

\documentclass[9pt]{developercv} % Default font size, values from 8-12pt are recommended
\usepackage{multicol}
\setlength{\columnsep}{0mm}
\usepackage[none]{hyphenat}
\raggedright
%----------------------------------------------------------------------------------------
\usepackage{lipsum}  



\begin{document}

%----------------------------------------------------------------------------------------
%	TITLE AND CONTACT INFORMATION
%----------------------------------------------------------------------------------------

\begin{minipage}[t]{0.5\textwidth} 
	\vspace{-\baselineskip} % Required for vertically aligning minipages
	
	{ \fontsize{16}{20} \textcolor{black}{\textbf{\MakeUppercase{Jacob Inwald}}}} % First name
	
	\vspace{6pt}
	
	{\Large Computer Science Student} % Career or current job title
\end{minipage}
\hfill
\begin{minipage}[t]{0.2\textwidth} % 20% of the page width for the first row of icons
	\vspace{-\baselineskip} % Required for vertically aligning minipages
	
	% The first parameter is the FontAwesome icon name, the second is the box size and the third is the text
	\icon{Globe}{11}{coming soon}\\ 
    \icon{Phone}{11}{+4407717386378}\\
    \icon{MapMarker}{11}{London, UK}\\
	
\end{minipage}
\begin{minipage}[t]{0.27\textwidth} % 27% of the page width for the second row of icons
	\vspace{-\baselineskip} % Required for vertically aligning minipages
	
	\icon{Envelope}{11}{\href{mailto:inwald.jacob@gmail.com}{inwald.jacob@gmail.com}}\\	
    \icon{Github}{11}{\href{https://github.com/JacobInwald}{github.com/JacobInwald}}\\
    \icon{LinkedinSquare}{11}{\href{https://www.linkedin.com/in/jacob-inwald-6b7889268}{/in/jacob-inwald-6b7889268}}\\    
    
\end{minipage}


%----------------------------------------------------------------------------------------
%	INTRODUCTION, SKILLS AND TECHNOLOGIES
%----------------------------------------------------------------------------------------

\begin{minipage}[t]{0.66\textwidth}
    \cvsect{Summary}
    From a young age, I have sought out problems to solve. 
	I got my first Raspberry Pi at 9 and I have not stopped programming since, learning C++ and Java to create games and Python to sketch out programming concepts. 
	I find that problem-solving is the most enjoyable and stimulating way to apply myself and I believe that the diversity of my project focuses shows this. 
    I excel in high-pressure and dynamic environments, as I have had to vie for excellence in many such environments - evidenced by my competitive experience in climbing.
\end{minipage}
\hfill % Whitespace between
\begin{minipage}[t]{0.265\textwidth}
    \cvsect{Skills}
    \vspace{-6pt}
    
    \begin{minipage}[t]{0.23\textwidth}
        \textbf{Languages:}
    \end{minipage}
    \hfill
    \begin{minipage}[t]{0.60\textwidth}
      Java, Python, C\#, C, Haskell, Scala, MIPS, TeX
    \end{minipage}
    \vspace{4mm}
    
    \begin{minipage}[t]{0.15\textwidth}
        \textbf{Software:}
    \end{minipage}
    \hfill
    \begin{minipage}[t]{0.68\textwidth}
      Wireshark, Visual Studio Code, Eclipse/IntelliJ, ...
    \end{minipage}
    
\end{minipage}


%----------------------------------------------------------------------------------------
%	EDUCATION
%----------------------------------------------------------------------------------------
\vspace{-10 pt}
\cvsect{Education}
\begin{entrylist}
    \entry
		{2021 - 2025}
		{BSc (Hons) Computer Science - predicted 1st}
		{University of Edinburgh}
		{Modules include: Computer Security, Reasoning and Agents, and Machine Learning}
    \entry
		{2019 - 2021}
		{Secondary Education}
		{JCoSS}
		{4 A-Levels at grade A* in Mathematics, Further Mathematics, Computer Science, and Physics}
	\entry
		{2017 - 2019}
		{10 GSCEs at grade A* and A** equivalent}
		{JCoSS}
		{ }
\end{entrylist}


%----------------------------------------------------------------------------------------
%	Projects
%----------------------------------------------------------------------------------------
\cvsect{Projects}
\begin{entrylist}
    \entry
		{Hash Cracking}
		{pwdtools}
		{\href{https://github.com/JacobInwald/pwdtools}{github.com}}
		{
			A library of password-related tools, from cracking plaintext passwords to cracking hashes. 
		 	% I also expanded this to allow for encryption/decryption using passwords as well as public key cryptography.
			I also included a GPU-accelerated brute-force password cracker, achieving a rate of 4 billion passwords tested a second. 
			\newline
			\texttt{python} \slashsep \texttt{metals-framework} \slashsep \texttt{Obj-C}
		}
    \entry
		{Dynamic Programming}
		{Seam Carving}
		{\href{https://github.com/JacobInwald/Seam-Carving}{github.com}}
		{
			I wrote a program to implement the seam-carving technique defined by Avidan and Shamir (2012). 
		 	It was interesting to apply the concept of dynamic programming to a concrete example.
			 \newline
			 \texttt{python}
		}
	\entry
		{Image Manipulation}
		{Genetic Image Creation}
		{\href{https://github.com/JacobInwald/generative-image-making}{github.com}}
		{
			I wrote a program that used an evolutionary algorithm to generate an image. 
			It created generations of quadrilaterals and then bred them together to find the best fit for the image.
			It created interesting, stylized images, which was my goal. 
			\newline
			\texttt{python}
		}
    \entry
		{Classification}
		{Optical Character Recognition - A-Level Project (achieved 98\%)}
		{\href{https://github.com/JacobInwald/OCRAlgebra}{github.com}}
		{
			I wrote a neural network that achieved 90\% accuracy on the MNIST (Modified National Institute of Standards and Technology) character set. 
			To make this a challenge, I did not allow myself to use any external libraries, apart from one for random number generation.  
			\newline
			\texttt{python}
		}
\end{entrylist}


%----------------------------------------------------------------------------------------
%	EXPERIENCE
%----------------------------------------------------------------------------------------
\vspace{-10 pt}
\cvsect{Experience}
\begin{entrylist}
	\entry
	{7/2022 -- 7/2022}
	{Work Experience – Software design}
	{HELIX Centre, Imperial College London, London W2}
	{
		I was involved in projects, coding a program to quantify differences between medical guidelines.
		I had to think critically and learn and adapt to complex tools in a short period, this included becoming proficient with the SNOMED database, made by the NHS. 
		\newline
		\texttt{python} \slashsep \texttt{database}
	}
	\entry
        {2019}
		{Google Code In}
		{Google}
		{
			I participated in the code-in where I had to complete various tasks for different companies. 
			This ranged from creating boot-up scripts for Fedora to writing a simple zipcracker. 
			This gave me some insight into how open-source projects worked properly. 
        	\newline
			\texttt{python} \slashsep \texttt{bash} \slashsep \texttt{open-source development}
		}
		
	% Maybe put on cyber-first
	% \entry
	% 	{2023 -- current \\\footnotesize{bursary holder}}
	% 	{CyberFirst Bursary Scheme}
	% 	{GCHQ}
	% 	{\vspace{-10pt}
    %     <insert info here>
    %   }
\end{entrylist}


%----------------------------------------------------------------------------------------
%	EMPLOYMENT
%----------------------------------------------------------------------------------------
\vspace{-10 pt}
\cvsect{Employment}
\begin{entrylist}
	\entry
        {7/2023 --  9/2023}
		{Bartender and Waiter}
		{Greenmantle, Greenmantle Group, Edinburgh EH8}
		{
			I ran a one-man box bar during the Fringe Festival in Edinburgh. 
			This was located directly outside the ZOO playground venue and therefore had many customers.
			My role was to manage this bar, maintain the correct levels of stock, ensure that the customer experience was as good as possible, to close and open the box bar.
			This required initiative as well as good management of resources.
			After the Fringe was over, I worked as a bartender at other pubs part of the Greenmantle Group, which required an ability to adapt quickly to new environments as well the ability to mesh well with new colleagues and co-workers.
        	\newline
			\texttt{management} \slashsep \texttt{teamwork} \slashsep \texttt{high-pressure}
		}
	\entry
		{8/2022 -- 9/2022}
		{Bartender and Waiter}
		{The Shakespeare, Greene King Pubs, Edinburgh EH3}
		{
			I worked during the Edinburgh Festival, serving both food and beverages.
			I had to mix drinks, change kegs, serve food, and provide a customer experience at the standard of Greene King Pubs.
			This role involved quick thinking, teamwork, and customer service, all in a high-pressure environment as it was during the international Fringe festival.
			\newline
			\texttt{customer service} \slashsep \texttt{teamwork} \slashsep \texttt{high-pressure}
		}
\end{entrylist}

\vspace{-10 pt}
\cvsect{Volunteering}
\begin{entrylist}
	\entry
		{10/2023 -- current}
		{Volunteer FOH}
		{Shrub Coop Community Hub, Tollcross, Edinburgh EH3}
		{
			Serving coffee and creating a welcoming environment for people.
			\newline
			\texttt{customer service} \slashsep \texttt{community}
		}
	\entry
		{1/2018 -- 7/2018}
		{Teaching Assistant}
		{The Castle Climbing Centre, London N4}
		{
			I assisted in teaching climbing techniques and safety to children from ages 9-15 every Saturday.
			This role involved leading warm-ups, creating engaging activities, and managing classes of 6 children to ensure their safety.
			\newline
			\texttt{teaching}
		}
\end{entrylist}


%----------------------------------------------------------------------------------------
%	Awards and Certificates
%----------------------------------------------------------------------------------------
\vspace{-10 pt}
\cvsect{Awards}
\begin{entrylist}
	\entry
        {2021}
		{Climbing Wall Instructor Training}
		{Mountain Training}
		{ 
			I learnt how to teach climbing and safety to new climbers effectively, learning how to lead sessions well. 
			\newline
        	\texttt{teaching}
		}
	\entry
        {2019}
		{DofE Silver}
		{Duke of Edinburgh}
		{ 
		}
\end{entrylist}


%----------------------------------------------------------------------------------------
%	Hobbies
%----------------------------------------------------------------------------------------
\vspace{-10 pt}
\cvsect{Hobbies}
\begin{entrylist}
	\entry
        {2021 -- current}
		{Edinburgh University Climbing Team}
		{University of Edinburgh}
		{\vspace{-10pt}
        \begin{itemize}[noitemsep,topsep=0pt,parsep=0pt,partopsep=0pt, leftmargin=-1pt]
            \item Competed in British University Climbing Series 2022, 2023, and will be competing in the 2024 event. 
            \item Competed in National Universities Lead and Speed CLimbing Competition 2023. 
            \item Organized and ran coaching sessions for the Edinburgh University Mountaineering Club
        \end{itemize} 
        \texttt{high-pressure performance} \slashsep \texttt{team-work} \slashsep \texttt{teaching}}
	\entry
		{2019 -- 2021}
		{The Castle Competition Squad}
		{The Castle Climbing Centre}
		{\vspace{-10pt}
        \begin{itemize}[noitemsep,topsep=0pt,parsep=0pt,partopsep=0pt, leftmargin=-1pt]
            \item Competed in the national British Mountaineering Club Youth Climbing Series, coming 4th at one event.
            \item Volunteered as an instructor for the younger classes. 
        \end{itemize} 
        \texttt{high-pressure performance} \slashsep \texttt{team-work} \slashsep \texttt{teaching}}
	\entry
		{2013 -- current}
		{Guitar and Piano}
		{Self-Taught}
		{
			I have played guitar since I was 10 and taught myself jazz piano over lockdown. 
			I have found comfort in music and take pride in being able to express myself with it. 
        \newline
		\texttt{music}
		}
\end{entrylist}

\end{document}
