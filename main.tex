%%%%%%%%%%%%%%%%%%%%%%%%%%%%%%%%%%%%%%%%%
% Developer CV
% LaTeX Class
% Version 2.0 (12/10/23)
%
% This class originates from:
% http://www.LaTeXTemplates.com
%
% Authors:
% Omar Roldan
% Based on a template by  Jan Vorisek (jan@vorisek.me)
% Based on a template by Jan Küster (info@jankuester.com)
% Modified for LaTeX Templates by Vel (vel@LaTeXTemplates.com)
%
% License:
% The MIT License (see included LICENSE file)
%
%%%%%%%%%%%%%%%%%%%%%%%%%%%%%%%%%%%%%%%%%

%----------------------------------------------------------------------------------------
%	PACKAGES AND OTHER DOCUMENT CONFIGURATIONS
%----------------------------------------------------------------------------------------

\documentclass[9pt]{developercv} % Default font size, values from 8-12pt are recommended
\usepackage{multicol}
\setlength{\columnsep}{0mm}
%----------------------------------------------------------------------------------------
\usepackage{lipsum}  


\begin{document}

%----------------------------------------------------------------------------------------
%	TITLE AND CONTACT INFORMATION
%----------------------------------------------------------------------------------------

\begin{minipage}[t]{0.5\textwidth} 
	\vspace{-\baselineskip} % Required for vertically aligning minipages
	
	{ \fontsize{16}{20} \textcolor{black}{\textbf{\MakeUppercase{Jacob Inwald}}}} % First name
	
	\vspace{6pt}
	
	{\Large Computer Science Student} % Career or current job title
\end{minipage}
\hfill
\begin{minipage}[t]{0.2\textwidth} % 20% of the page width for the first row of icons
	\vspace{-\baselineskip} % Required for vertically aligning minipages
	
	% The first parameter is the FontAwesome icon name, the second is the box size and the third is the text
	\icon{Globe}{11}{\href{http://www.google.com}{coming soon}}\\ 
    \icon{Phone}{11}{+4407717386378}\\
    \icon{MapMarker}{11}{London, UK}\\
	
\end{minipage}
\begin{minipage}[t]{0.27\textwidth} % 27% of the page width for the second row of icons
	\vspace{-\baselineskip} % Required for vertically aligning minipages
	
	\icon{Envelope}{11}{\href{mailto:inwald.jacob@gmail.com}{inwald.jacob@gmail.com}}\\	
    \icon{Github}{11}{\href{https://github.com/JacobInwald}{github.com/JacobInwald}}\\
    \icon{LinkedinSquare}{11}{\href{https://www.linkedin.com/in/jacob-inwald-6b7889268}{/in/jacob-inwald-6b7889268}}\\    
    
\end{minipage}


%----------------------------------------------------------------------------------------
%	INTRODUCTION, SKILLS AND TECHNOLOGIES
%----------------------------------------------------------------------------------------

\begin{minipage}[t]{0.46\textwidth}
    \cvsect{Summary}
	\vspace{-6pt}
 
    %Dummy text
	I am a 3rd year computer science student ...  \\
\end{minipage}
\hfill % Whitespace between
\begin{minipage}[t]{0.465\textwidth}
    \cvsect{Skills}
    \vspace{-6pt}
    
    \begin{minipage}[t]{0.2\textwidth}
        \textbf{Languages:}
    \end{minipage}
    \hfill
    \begin{minipage}[t]{0.73\textwidth}
      Java, Python, C++, C, Haskell, Scala, 
    \end{minipage}
    \vspace{4mm}
    
    \begin{minipage}[t]{0.2\textwidth}
        \textbf{Technologies:}
    \end{minipage}
    \hfill
    \begin{minipage}[t]{0.73\textwidth}
      Wireshark, ...
    \end{minipage}
    
\end{minipage}

%----------------------------------------------------------------------------------------
%	Projects
%----------------------------------------------------------------------------------------
\cvsect{Projects}
\begin{entrylist}
    \entry
		{Hash Cracking}
		{pwdtools}
		{\href{https://github.com/JacobInwald/pwdtools}{github.com}}
		{
			A library of password related tools, from cracking plaintext passwords to cracking hashes. 
		 	I also expanded this to allow for encryption/decryption using passwords as well as public key cryptography.
		 }
    \entry
		{Dynamic Programming}
		{Seam Carving}
		{\href{https://github.com/JacobInwald/pwdtools}{github.com}}
		{
			I wrote a program to implement the seam-carving technique defined in Avidan and Shamir (2012). 
		 	It was interesting to apply the concept of dynamic programming to a concrete example.
		}
	\entry
		{Image Manipulation}
		{Genetic Image Create}
		{\href{https://github.com/JacobInwald/pwdtools}{github.com}}
		{
			I wrote a program that used an evolutionary algorithm to generate an image. 
			It created generations of quadrilaterals and then bred them together to find the best fit for the image.
			It created interesting, stylized images, which was my goal. 
		}
    \entry
		{Classification}
		{Optical Character Recognition - A-Level Project (achieved 98\%)}
		{\href{https://github.com/JacobInwald/pwdtools}{github.com}}
		{
			I wrote a neural network that achieved 90\% accuracy on the MNIST character set. 
			To make this a challenge, I did not allow myself to use any external libraries, apart from one for PRG.  
		}
\end{entrylist}

%----------------------------------------------------------------------------------------
%	EDUCATION
%----------------------------------------------------------------------------------------
\vspace{-10 pt}
\cvsect{Education}
\begin{entrylist}
    \entry
		{2021 - 2025}
		{BSc (Hons) Computer Science - predicted 1st}
		{University of Edinburgh}
		{Modules include: Computer Security, Reasoning and Agents, Machine Learning}
    \entry
		{2019 - 2021}
		{Secondary Education}
		{JCoSS}
		{4 A-Levels at grade A* in Mathematics, Further Mathematics, Computer Science, and Physics}
	\entry
		{2017 - 2019}
		{10 GSCEs at grade A* and A** equivalent}
		{JCoSS}
		{ }
\end{entrylist}

%----------------------------------------------------------------------------------------
%	EXPERIENCE
%----------------------------------------------------------------------------------------
\vspace{-10 pt}
\cvsect{Experience}
\begin{entrylist}
	\entry
        {x/2023 -- x/2023}
		{\lipsum[1][1]}
		{Company}
		{\vspace{-10pt}
        \begin{itemize}[noitemsep,topsep=0pt,parsep=0pt,partopsep=0pt, leftmargin=-1pt]
            \item \lipsum[1][1-2]
            \item \lipsum[1][3-4]
        \end{itemize} 
        \texttt{SQL} \slashsep \texttt{Excel}}
	\entry
		{x/2023 -- x/2023}
		{\lipsum[1][1]}
		{Company}
		{\vspace{-10pt}
        \begin{itemize}[noitemsep,topsep=0pt,parsep=0pt,partopsep=0pt, leftmargin=-1pt]
            \item \lipsum[1][1-2]
            \item \lipsum[1][3-4]
        \end{itemize} 
        \texttt{SQL} \slashsep \texttt{Excel}}
	\entry
		{x/2023 -- x/2023 \\\footnotesize{scholarship holder}}
		{\lipsum[1][1]}
		{Company}
		{\vspace{-10pt}
        \begin{itemize}[noitemsep,topsep=0pt,parsep=0pt,partopsep=0pt, leftmargin=-1pt]
            \item \lipsum[1][1-2]
            \item \lipsum[1][3-4]
        \end{itemize} 
        \texttt{SQL} \slashsep \texttt{Excel}}
\end{entrylist}


%----------------------------------------------------------------------------------------
%	Hobbies
%----------------------------------------------------------------------------------------
\vspace{-10 pt}
\cvsect{Hobbies}
\begin{entrylist}
	\entry
        {2021 -- current}
		{Edinburgh University Climbing Team}
		{University of Edinburgh}
		{\vspace{-10pt}
        \begin{itemize}[noitemsep,topsep=0pt,parsep=0pt,partopsep=0pt, leftmargin=-1pt]
            \item Competed in BUCS 2022, 2023, and will be competing in the 2024 event. 
            \item Competed in NULSCC 2023. 
            \item Organized and ran coaching sessions for the Edinburgh University Mountainnering Club
        \end{itemize} 
        \texttt{High-pressure performance} \slashsep \texttt{Team-work} \slashsep \texttt{Teaching}}
	\entry
		{2019 -- 2021}
		{The Castle Competition Squad}
		{The Castle Climbing Centre}
		{\vspace{-10pt}
        \begin{itemize}[noitemsep,topsep=0pt,parsep=0pt,partopsep=0pt, leftmargin=-1pt]
            \item Competed in the BMC Youth Climbing Series, coming 4th at one event.
            \item Volunteered as an instructor for the younger classes. 
        \end{itemize} 
        \texttt{High-pressure performance} \slashsep \texttt{Team-work} \slashsep \texttt{Teaching}}
	% Maybe put on cyber-first
	% \entry
	% 	{x/2023 -- x/2023 \\\footnotesize{scholarship holder}}
	% 	{\lipsum[1][1]}
	% 	{Company}
	% 	{\vspace{-10pt}
    %     \begin{itemize}[noitemsep,topsep=0pt,parsep=0pt,partopsep=0pt, leftmargin=-1pt]
    %         \item \lipsum[1][1-2]
    %         \item \lipsum[1][3-4]
    %     \end{itemize} 
    %     \texttt{SQL} \slashsep \texttt{Excel}}
\end{entrylist}

%----------------------------------------------------------------------------------------
%	LANGUAGES
%----------------------------------------------------------------------------------------
\vspace{-10 pt}
	\cvsect{Languages}
    \vspace{-6pt}
    
    \hspace{26mm} \textbf{English} - natives

%----------------------------------------------------------------------------------------

\end{document}
